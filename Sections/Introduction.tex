\section{Introduction} \label{sec:intro}

\subsection{Purpose} \label{subsec:purpose}
This document is called \emph{Requirements Analysis and Specification Document} also knowed as the acronym RASD. It's purpose is to communicate to customers what is understanding about functional and not-functional requirements based, the limitations and obstacles for implement this system, the constraints founded and for modeling the customer's need. This document is also addressed to developers and programmers who have to implements all the requirements then it must be more complete and correct than possible. It is a contract with customers therefore it must show use cases to allow everyone to understand what the system will do and in what domain it can be used. Project manager can use this document to make an evaluation of costs and size of the project.


\subsection{Scope} \label{subsec:scope}
The aim of the project called PowerEnJoy is to provide a car-sharing service that involves only electric cars. All people who wants to share a car must be able to register at the system using credentials such as name, surname, e-mail, nickname and giving a valid information payment (number of credit card) that is needed to pay for the service. When the user receive the password to log in, he can find available cars in a specific location and, if he want, he can reserve it. The system unlocks the car as soon as the user is nearby and keep informed of the amount of the service with a screen on the car; when the car is in one of the safe areas, indicated in a list that can be consulted on-line, the system locks it after the person exit from it. The project has also the purpose to encouraged people to left at home their pollutants cars and takes with other people the electric car: in fact if there are at least two passengers with the driver he has a 10 per cent discount on the ride. If the user left the car in the safe area with more than half of the battery he has a 20 per cent discount on the ride and if he recharged the car he will have 30 per cent discount. But if the user left the car far away from a safe area or with more than 80 per cent of empty battery, the system charge 30 per cent more on the ride.

%Identifies the product and application domain
%Analysis of the world and of the shared phenomena
\subsection{Actors} \label{subsec:actors}
\begin{itemize}
\item VISITOR: the person who visits the systems but that are not log-in in the site, he can only see the home page, the page with the form for the registration, where he must provide all the requested information, and he has the possibility to log-in with the password given by the system when the registration is successfully committed. 
\item USER: the person who has successfully log-in, he can do all the operations provided by the system through the user interface such as reserve a car or consult the list of available cars or say to the system that he is nearby the reserved car. 
\end{itemize}

\subsection{Goals} \label{subsec:goals}
\begin{itemize}
\item The system must be able to allow the visitor to register giving to him the correct form and verifying the correctness of the information that are provided by him.
\item The system must be able to allow the visitor to log-in, first of all checking if the user is registered and after comparing username and password written in the form with the ones saved in the database: only if all the information are correct the user can access to the system, otherwise the page said there is an error and what type of error is.
\item The system must be able to provide to the user the list of available cars near his position or a specific location.
\item The system must be able to allow the user to reserve a car up to one hour before it is picked up.
\item The system must be able to know when the user is nearby the reserved car and unlock it; as soon as the engine ignites, the system started charging the user and keep informed the user trough a display in the car.
\item If the user not pick up the car after one hour from the registration, the system delete the registration giving a fee of 1 euro to the user and make the car available.

MANCANO ALTRI GOAL

\end{itemize}


\subsection{Definitions, acronyms, abbreviations} \label{subsec:def}

Here there is the acronims list:

%------------Acronyms----------
\begin{acronym}[RASD] %put here the first acronym

\acro{rads}[RASD]{Requirements Analysis and Specifications Document}
%lists of used acronyms

\end{acronym}


\subsection{Overview} \label{subsec:overview}

%Describes contents and structure of the remainder of the RASD

