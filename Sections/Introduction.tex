\section{Introduction} \label{sec:intro}

\subsection{Purpose} \label{subsec:purpose}
This document is called \emph{Requirements Analysis and Specification Document} also knowed as the acronym RASD. It's purpose is to communicate to customers what is understanding about functional and not-functional requirements based, the limitations and obstacles for implement this system, the constraints founded and for modeling the customer's need. This document is also addressed to developers and programmers who have to implements all the requirements then it must be more complete and correct than possible. It is a contract with customers therefore it must show use cases to allow everyone to understand what the system will do and in what domain it can be used. Project manager can use this document to make an evaluation of costs and size of the project.


\subsection{Scope} \label{subsec:scope}
The aim of the project called PowerEnJoy is to provide a car-sharing service that involves only electric cars. All people who wants to share a car must be able to register at the system using credentials such as name, surname, e-mail, nickname and giving a valid information payment (number of credit card) that is needed to pay for the service. When the user receive the password to log in, he can find available cars in a specific location and, if he want, he can reserve it. The system unlocks the car as soon as the user is nearby and keep informed of the amount of the service with a screen on the car; when the car is in one of the safe areas, indicated in a list that can be consulted on-line, the system locks it after the person exit from it. The project has also the purpose to encouraged people to left at home their pollutants cars and takes with other people the electric car: in fact if there are at least two passengers with the driver he has a 10 per cent discount on the ride. If the user left the car in the safe area with more than half of the battery he has a 20 per cent discount on the ride and if he recharged the car he will have 30 per cent discount. But if the user left the car far away from a safe area or with more than 80 per cent of empty battery, the system charge 30 per cent more on the ride.

%Identifies the product and application domain
%Analysis of the world and of the shared phenomena

\subsection{Definitions, acronyms, abbreviations} \label{subsec:def}

Here there is the acronims list:

%------------Acronyms----------
\begin{acronym}[RASD] %put here the first acronym

\acro{rads}[RASD]{Requirements Analysis and Specifications Document}
%lists of used acronyms

\end{acronym}


\subsection{Overview} \label{subsec:overview}

%Describes contents and structure of the remainder of the RASD

