\section{Overall Description} \label{sec:description}
In this section, we will explain the product prospective and major functionality and also the characteristic of the user that we think he will use this application. Moreover, we will show all the constraints we have found and all the assumptions that we have made in order to make the requirements clear in all their parts. 

\subsection{Product perspective} \label{subsec:prod_persp}
Our application is a self-contained product based on a client interface, that helps the user to interact with this system. The application also uses an \acs{api} for the geolocation provided by \emph{\textbf{Google Maps}} because we need to know both the cars and user's positions. It is supported by common browser such as "\emph{Chrome}". It is also closely linked with a \acs{db} that contains all users and cars data, the list of safe areas and if there is a power grid station in there. Because of our assumptions, the system must collaborate with the motorization's system to check user's driving license and with an external society which provides the  insurance for the cars. The system must also collaborate with a Payment Service Provider that checks all the payment information and makes transaction secure both for the user and for the software's owner.

%Describes external interfaces: system, user, hardware, software; also operations and site adaptation, and hardware constraints

%Further details on the shared phenomena and a domain model (class) diagrams, and statecharts

\subsection{Product functions} \label{subsec:prod_funct}

%Summary of major functions, requirements

\subsection{User characteristics} \label{subsec:user_char}
The user we expect will use our application is a person that wants to rent a car near his position, whenever he needs, in an efficient and rapid way; he is also a person who is interested in the problem of pollution since our system involves only electric car. Our application doesn't require a particular ability in using IT equipment: in fact, thanks to the friendly user interface, he must only be able to do simple actions, such as insert his credential for registration, log in and eventually set the position from where he wants to share the car. Thanks to the position, the system will do all the operation to lock and unlock the car and charge the user.

%Anything that is relevant to clarify their needs

\subsection{Constraints} \label{subsec:constraints}
Our software must be used at the same time by a lot of users. It doesn't have to consume a lot of battery and memory since it will be used also on mobile. 
For the complete development and functionality of our product, we need to know the position of the user. In order to do that, the user, when registering at our service, must agree with our policy of privacy; otherwise, he won't be able to use our software.
%The policy of privacy is very important: in fact, if the user doesn't sign and approve the warranty of privacy, the system can't be used. Morover, it is a constraint for the system development because for the use of the application we must know the position of the user and that can't be done without the approval of the user. 
Another constraint is that, since we use a Google \acs{api}, we must follow its regulation and developer guide. We must also follow the PCI Security Standard Council because we manage payment actions.

%Anything that will limit the developer’s options (e.g. regulations, reliability, criticality, hardware limitations, parallelism, etc)

\subsection{Assumptions and Dependencies} \label{subsec:dependencies}
\begin{enumerate}
\item There isn't a privileged user: all the users can do the same things and there isn't a limitation in system's use.
\item The visitors can only see the homepage, the registration's form and the log in page.
\item There isn't dependency from users.
\item A user can't reserve two cars at the same hour at the same day: he can reserve only one car each time.
\item There is a user's page with the payment history: so, the user can see all details for a specific payment because all the operation must be clear. This is a warranty of correctness between system's owner and users.
\item When a reservation is deleted because one hour is passed, the user is informed trough a notification like this: \emph{"The reservation for car X at the hour Y for the day Z is deleted because is passed one hour. You must pay a fee of 1 euro"}.
\item In the list of available cars is specified the car model with all the details so the user can select the best one for his needs. It is also indicated whether the vehicle is suitable for the transport of disabled persons.
\item During the reservation there is a remainder to know when the reservation will expire.
\item In the site there is a list of possible discount and its details.
\item During the registration we ask the user to insert his driving license's number: in this way the system can verify if it is valid or not.
\item The system provides the opportunity to receive trough the e-mail the password, if the user doesn't remember it.
\item The car's insurance is provided by an external society whereby we have stipulate a contract.
\item All the payment actions are managed by an external Payment Service Provider.
\end{enumerate}
%domain assumptions

