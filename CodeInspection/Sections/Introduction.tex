\section{Introduction} \label{sec:intro}

Code inspection is the systematic examination of computer source code. Our scope in this phase is to find mistakes overlooked during the initial development phase, in order to improve the quality of the software and also the developers' skill.\newline
Moreover, this technique is one of the most effective to identify security flaws. 
In this specific context, the manual approach has a great value: a human reviewer can understand the context for coding practices and, consequently, he can make a valid risk estimation, since he has the ability to evaluate both the probability of attack and the business impact that a breach could have on the company.

\subsection{Purpose}

We are now going to analize the purpose of the code inspection. \newline
The first purpose is to check that the code of the software has at least the quality needed in order to be released. When reviewing the code, the developers are able to find errors of all types, for example those that derives from a poor structure, but also the simpler one caused by omissions. \newline
The second purpose is to help developers improve they ability about when and how to apply techniques in order to improve code quality, consistency, and maintainability.

\subsection{List of Definitions and Abbreviations}
Here there is the acronims and abbreviations list:

%------------Acronyms----------

\begin{acronym}[EOF] %put here the first acronym

\acro{eof}[EOF]{End Of File}

\end{acronym}

\subsection{List of Reference Documents}

\begin{itemize}
\item[\textbf{--}] Specification document: Code Inspection Assignment Task Description.pdf
\item[\textbf{--}] \textit{Oracle Java Code Convention}: \url{http://www.oracle.com/technetwork/java/javase/documentation/codeconvtoc-136057.html}
\end{itemize}

\subsection{Document overview}

Here we show the stucture of the document, with a brief overview of each section.

\begin{itemize}

\item[\textbf{Section \ref{sec:intro}}]There is an introduction with general information about this document.

\item[\textbf{Section \ref{sec:code}}]There are shown the coding convention and the issue checklist used for the code inspection.

\item[\textbf{Section \ref{sec:code-proc}}]There is the central part, in which we declare the classes of code to be inspected and the issue found.

\item[\textbf{Section \ref{sec:app}}]Here there are given additional information that may be useful to the reader, such as the tools used and the time spent to redact this document.
\end{itemize}