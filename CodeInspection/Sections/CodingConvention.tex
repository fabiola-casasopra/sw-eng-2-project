\section{Coding Conventions and Issue Checklist} \label{sec:code}

\subsection{Coding Conventions}
Coding conventions are a set of guidelines for a specific programming language that recommend programming style, practices and methods for each aspect of
a program written in this language. These conventions are useful for the developers, since they allow to improve the readability of their source code and make software maintenance easier. The convention are guidelines for software structural quality, since they are about file organization, indentation, comments, declarations, statements, white space, naming conventions, programming practices, programming principles, programming rules of thumb, architectural best practices, and more on. 
Coding conventions cannot be checked by a compliler, but only by human maintainers and peer reviewers of a software project. 
\newline

As far as our code inspection is concerned, we are going to follow the \textit{Oracle Java Code Conventions}.

\subsection{Issue Checklist}
In this part of the document, we are going to report the checklist of possible issues that we are going to refer during the inspection process.

\subsubsection{Naming Convention} \label{sssec:naming}

\begin{enumerate}

\item All class names, interface names, method names, class variables, method variables, and constants used should have meaningful names and do what the name suggests.

\item If one-character variables are used, they are used only for temporary "throwaway" variables, such as those used in for loops.

\item Class names are nouns, in mixed case, with the first letter of each word in capitalized. Examples: \texttt{class Raster}; \texttt{class ImageSprite};

\item Interface names should be capitalized like classes.

\item Method names should be verbs, with the first letter of each addition word
capitalized. Examples: \texttt{getBackground()}; \texttt{computeTemperature()}.

\item Class variables, also called attributes, are mixed case, but might begin with an underscore ('$\_$') followed by a lowercase first letter. All the remaining words in the variable name have their first letter capitalized. Examples: \texttt{$\_$windowHeight}, \texttt{timeSeriesData}.

\item Constants are declared using all uppercase with words separated by an underscore. Examples: \texttt{MIN$\_$WIDTH}; \texttt{MAX$\_$HEIGHT}.

\end{enumerate}

\subsubsection{Indention} \label{sssec:indention}

\begin{enumerate}
\setcounter{enumi}{7}
  
\item Three or four spaces are used for indentation and done so consistently.
  
\item No tabs are used to indent.

\end{enumerate}

\subsubsection{Braces} \label{sssec:braces}

\begin{enumerate}
\setcounter{enumi}{9}

\item Consistent bracing style is used, either the preferred "Allman" style (first brace goes underneath the opening block) or the "Kernighan and Ritchie" style (first brace is on the same line of the instruction that opens the new block).

\item All \texttt{if}, \texttt{while}, \texttt{do-while}, \texttt{try-catch}, and \texttt{for} statements that have only one statement to execute are surrounded by curly braces. Example: avoid this:

\begin{lstlisting}
        if ( condition )
            doThis();
\end{lstlisting}

instead do this:
\begin{lstlisting}
        if ( condition )
        {
            doThis();
        }
\end{lstlisting}

\end{enumerate}

\subsubsection{File Organization} \label{sssec:file-org}

\begin{enumerate}
\setcounter{enumi}{11}

\item Blank lines and optional comments are used to separate sections (begin- ning comments, package/import statements, class/interface declarations which include class variable/attributes declarations, constructors, and methods).

\item Where practical, line length does not exceed 80 characters.

\item When line length must exceed 80 characters, it does NOT exceed 120 characters.

\end{enumerate}

\subsubsection{Wrapping Lines} \label{sssec:wrap}

\begin{enumerate}
\setcounter{enumi}{14}

\item Line break occurs after a comma or an operator.

\item Higher-level breaks are used.

\item A new statement is aligned with the beginning of the expression at the same level as the previous line.

\end{enumerate}

\subsubsection{Comments} \label{sssec:comm}

\begin{enumerate}
\setcounter{enumi}{17}

\item Comments are used to adequately explain what the class, interface, meth- ods, and blocks of code are doing.

\item Commented out code contains a reason for being commented out and a date it can be removed from the source file if determined it is no longer needed.

\end{enumerate}

\subsubsection{Java Source File} \label{sssec:source}

\begin{enumerate}
\setcounter{enumi}{19}

\item Each Java source file contains a single public class or interface.

\item The public class is the first class or interface in the file.

\item Check that the external program interfaces are implemented consistently with what is described in the javadoc.

\item Check that the javadoc is complete (i.e., it covers all classes and files part of the set of classes assigned to you).

\end{enumerate}

\subsubsection{Package and Import Statements} \label{sssec:import}

\begin{enumerate}
\setcounter{enumi}{23}

\item If any package statements are needed, they should be the first non- comment statements. Import statements follow.

\end{enumerate}

\subsubsection{Class and Interface Declarations} \label{sssec:decl}

\begin{enumerate}
\setcounter{enumi}{24}

\item The class or interface declarations shall be in the following order:

	\begin{enumerate}
	
	\item class/interface documentation comment;
	
	\item class or interface statement;

	\item class/interface implementation comment, if necessary;
	
	\item class (static) variables;
	
		\begin{enumerate}
			
			\item  first public class variables;
			
			\item next protected class variables;
			
			\item next package level (no access modifier);
			
			\item next package level (no access modifier);			
			
		\end{enumerate}

	\item instance variables;
	
		\begin{enumerate}
		
			\item first public instance variables;
			
			\item next protected instance variables;
			
			\item next package level (no access modifier);
			
			\item last private instance variables.		
		
		\end{enumerate}

	\item constructors;
	
	\item methods.

	\end{enumerate}
	
\item Methods are grouped by functionality rather than by scope or accessi- bility.

\item Check that the code is free of duplicates, long methods, big classes, break- ing encapsulation, as well as if coupling and cohesion are adequate.

\end{enumerate}

\subsubsection{Initialization and Declarations} \label{sssec:init}

\begin{enumerate}
\setcounter{enumi}{27}

\item Check that variables and class members are of the correct type. Check that they have the right visibility (public/private/protected).

\item Check that variables are declared in the proper scope.

\item Check that constructors are called when a new object is desired.

\item Check that all object references are initialized before use.

\item Variables are initialized where they are declared, unless dependent upon a computation.

\item Declarations appear at the beginning of blocks (A block is any code surrounded by curly braces ('$\{$' and '$\}$'). The exception is a variable can be declared in a for loop.

\end{enumerate}

\subsubsection{Method Calls} \label{sssec:calls}

\begin{enumerate}
\setcounter{enumi}{33}

\item Check that parameters are presented in the correct order.

\item Check that the correct method is being called, or should it be a different method with a similar name.

\item Check that method returned values are used properly.

\end{enumerate}

\subsubsection{Arrays} \label{sssec:arrays}

\begin{enumerate}
\setcounter{enumi}{36}

\item Check that there are no off-by-one errors in array indexing (that is, all required array elements are correctly accessed through the index).

\item Check that all array (or other collection) indexes have been prevented from going out-of-bounds.

\item Check that constructors are called when a new array item is desired.

\end{enumerate}

\subsubsection{Object Comparison} \label{sssec:obj-comp}

\begin{enumerate}
\setcounter{enumi}{39}

\item Check that all objects (including Strings) are compared with equals and not with \texttt{$==$}.

\end{enumerate}

\subsubsection{Output Format} \label{sssec:output}

\begin{enumerate}
\setcounter{enumi}{40}

\item Check that displayed output is free of spelling and grammatical errors.

\item Check that error messages are comprehensive and provide guidance as to how to correct the problem.

\item Check that the output is formatted correctly in terms of line stepping and spacing.

\end{enumerate}

\subsubsection{Computation, Comparisons and Assignments} \label{sssec:cca}

\begin{enumerate}
\setcounter{enumi}{43}

\item  Check that the implementation avoids "brutish programming": (see \url{http://users.csc.calpoly.edu/~jdalbey/SWE/CodeSmells/bonehead.html}).

\item Check order of computation/evaluation, operator precedence and paren- thesizing.

\item Check the liberal use of parenthesis is used to avoid operator precedence problems.

\item Check that all denominators of a division are prevented from being zero.

\item Check that integer arithmetic, especially division, are used appropriately to avoid causing unexpected truncation/rounding.

\item Check that the comparison and Boolean operators are correct.

\item  Check throw-catch expressions, and check that the error condition is actually legitimate.

\item Check that the code is free of any implicit type conversions.

\end{enumerate}

\subsubsection{Exceptions} \label{sssec:exce}

\begin{enumerate}
\setcounter{enumi}{51}

\item Check that the relevant exceptions are caught.

\item  Check that the appropriate action are taken for each catch block.

\end{enumerate}

\subsubsection{Flow of Control} \label{sssec:flow}

\begin{enumerate}
\setcounter{enumi}{53}

\item In a \texttt{switch} statement, check that all cases are addressed by break or return.

\item Check that all \texttt{switch} statements have a \texttt{default} branch.

\item Check that all loops are correctly formed, with the appropriate initial- ization, increment and termination expressions.

\end{enumerate}

\subsubsection{Files} \label{sssec:files}

\begin{enumerate}
\setcounter{enumi}{56}

\item Check that all files are properly declared and opened.

\item Check that all files are closed properly, even in the case of an error.

\item Check that \acs{eof} conditions are detected and handled correctly.

\item Check that all file exceptions are caught and dealt with accordingly.

\end{enumerate}