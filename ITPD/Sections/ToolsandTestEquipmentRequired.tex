\section{Tools and Test Equipment Required} \label{sec:tter}
%Identify all tools and test equipment needed to accomplish the integration. Refer to the tools presented during the lectures. Explain why and how you are going to use them. Note that you may also use manual testing for some part. Consider manual testing as one of the possible tools you have available.

In this section of the document, we are going to identify all tools and test equipment needed to accomplish the integration and to explain why we are going to use them.
In order to carry out the \acl{it}, we think that the best solution is to use \texttt{Arquillian} (more info at \url{http://arquillian.org}). It is an integration testing used to execute test cases against the container: interactions with the system are as important as the performed work. Moreover, an Arquillian test it is not complex, because it looks just like a \textit{JUnit test}, but with some more functionalities. Another positive aspect is that its framework is compatible with JEE containers, that are the ones on which this project relies on.
\newline

Furthermore, we think that a tool such as Jmeter (more info at \url{http://jmeter.apache.org}) can be useful. It allows us to load test functional behavior and measure performance. It can be used to simulate a heavy load on a server, network, or object, to test its strength or to analyze overall performance under different load types. This can be a good way to verify the scalability of our application for a large number of user.
\newline

In addition to what described above, a \acs{gps} Receiver and a smartphone (or a tool to simulate them) are needed, with characteristics that respect all the requirements we have already defined in the \acs{rasd}.