\section{Introduction} \label{sec:intro}

The main purpose of this document and the project's scope are described in this section. Moreover, we are going to give some definitions which will help the reader to better understand the content of this document and we are going to show the reference documents that have been used to redact this one. At the end, it will be explained the structure of this document.

\subsection{Purpose} \label{subsec:purpose}

This document is called \emph{\acl{dd}} and, from now on, we will refer to it using the acronym \acs{dd}. Its purpose is to provide a complete description of the system specified in the \acs{rasd}, giving enough technical details to allow the proceeding of the software development. So, we must have a good understanding of which are the components of the system, how they interact, which is their high level architecture and how they will be deployed, highlighting the design patterns we decided to use.
In addition, the \acs{rasd} is useful for the developers in order to figure out how to implement the entire system, thanks to the general description of the architecture and the design of the system to be built. For this reason, it must be complete and correct as much as possible.
The \acs{dd} contains both narrative and graphical documentation of the software design, including, for example, user experience diagrams, entity-relation diagrams, component diagrams, and other supporting requirement information.

\subsection{Scope} \label{subsec:scope}
The aim of the project PowerEnJoy is to provide a car-sharing service that involves \textit{only} electric cars. In this document, we will give more information about the design choices for the development of this application.
In order to have more details about the scope of our project, you may refer to the \textit{Section 1} of the \acl{rasd}.

\subsection{Definitions, acronyms, abbreviations} 
\label{subsec:def-ac}
\subsubsection{Definitions} \label{def}
Here we present some fundamental defitinition:
\begin{itemize}
\item[\textbf{--}] A \textsf{Session Bean} encapsulates business logic that can be invoked programmatically by a client over local, remote, or web service client views. To access an application deployed on the server, the client invokes the session bean's methods. The session bean performs work for its client, shielding it from complexity by executing business tasks inside the server. Moreover, a session bean is not persistent. 

\item[\textbf{--}] A \textsf{Stateful Session Bean} is a session bean in which the instance variables (that is the state of the object) represent the state of a unique client/bean session.
A session bean is not shared: it can have only one client per time. When the client terminates, its session bean appears to terminate and is no longer associated with it.
The state is retained only for the duration of the client/bean session. If the client removes the bean, the session ends and the state disappears. This transient nature of the state is not a problem, however, because when the conversation between the client and the bean ends, there is no need to retain the state.

\item[\textbf{--}] A \textsf{Stateless Session Bean} does not maintain a conversational state with the client. When a client invokes the methods of a stateless bean, the bean's instance variables may contain a state specific to that client but only for the duration of the invocation. When the method is finished, the client-specific state should not be retained. 
Because they can support multiple clients, stateless session beans can offer better scalability for applications that require large numbers of clients.

\item[\textbf{--}] A \textsf{Singleton Session Bean} is instantiated once per application and exists for the lifecycle of the application. Singleton session beans are designed for circumstances in which a single enterprise bean instance is shared across and concurrently accessed by clients.

\item[\textbf{--}] \textsf{\acl{jsf}} technology is a server-side component framework for building Java technology-based web applications.
\acs{jsf} technology provides a well-defined programming model and various tag libraries. The tag libraries contain tag handlers that implement the component tags. These features significantly ease the burden of building and maintaining web applications with server-side \acs{ui}s.

\end{itemize}

These definitions are taken from "\textit{Java Platform, Enterprise Edition The Java EE Tutorial, Release 7}", released by Oracle.

\subsubsection{Acronyms and abbreviations } \label{acr}
Here there is the acronims and abbreviations list:

%------------Acronyms----------
\begin{acronym}[DD] %put here the first acronym

\acro{dd}[DD]{Design Document}
\acro{gui}[GUI]{Graphical User Interface}
\acro{eis}[EIS]{Enterprise Information System}
\acro{ejb}[EJB]{Enterprise JavaBean}
\acro{er}[ER]{Entity-Relationship}
\acro{gps}[GPS]{Global Positioning System}
\acro{jdbc}[JDBC]{Java DataBase Connectivity}
\acro{jee}[JEE]{Java Platform, Enterprise Edition}
\acro{jsf}[JSF]{JavaServer Faces}
\acro{mvc}[MVC]{Model-View-Controller}
\acro{rasd}[RASD]{Requirements Analysis and Specifications Document}
\acro{rest}[REST]{Representational State Transfer}
\acro{ui}[UI]{User Interface}
\acro{uml}[UML]{Unified Modeling Language}
%\acro{api}[API]{Application Programming Interface}
%\acro{db}[DB]{DataBase}
%\acro{os}[OS]{Operating System}

\end{acronym}

\subsection{Reference Documents} \label{ref-doc}

\begin{itemize}
\item[\textbf{--}] Specification document: Assignments AA 2016-2017.pdf
\item[\textbf{--}] IEEE Std 1016tm-2009 Standard for Information Technology - System Design - Software Design Descriptions.
\item[\textbf{--}] \acl{rasd}: RASD.pdf 
\newline
(\url{https://github.com/fabiola-casasopra/sw-eng-2-project/tree/master/RASD/RASD.pdf})
\end{itemize}

\subsection{Document Structure} \label{doc-struct}

Here is presented the stucture of the document,  with a brief overview of each section.
While the \acs{rasd} is written for a more general audience, this document is intended for only people directly involved in the development of our application, as the software developers, the project consultants and the team managers. 
So, each person, depending on its role, can go directly and read to the section he finds more relevant. 

\begin{itemize}

\item[\textbf{Section \ref{sec:intro}}]There is an introduction with this document's purpose and other general information about it.

\item[\textbf{Section \ref{sec:arch-design}}]There is an overall view of our system, describing all the components from different points of view and  highlighting their interaction. Moreover, there is an explanation about the selected architectural system and design pattern.

\item[\textbf{Section \ref{sec:algo}}]Here there are presented the algorithm we think are more relevant for the development of the application. They are mainly described using a pseudocode implementation.

\item[\textbf{Section \ref{sec:user-interface}}]There is a description of all the details about the structure of the \acl{gui}. This section is useful for the reder  to get an idea on how the final application will look like.

\item[\textbf{Section \ref{sec:req-trac}}]There is an explaination of how the requirements defined in the \acs{rasd} map into the design elements that have defined in this document.

\item[\textbf{Section \ref{sec:appendix}}]Here there are given additional information that may be useful to the reader, such as the tools used and the time spent to redact this document.
\end{itemize}

