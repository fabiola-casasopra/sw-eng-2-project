\section{Architectural Design} \label{sec:arch-design}

In this section, we will show the proposed architecture for our system, that allow us to give a more complete and general idea of the entire system and a better view of the relation with external components.

\subsection{Overview} \label{subsec:overview}
Now we are going to present an overall description of the architecture of our system. We propose a 4-tier architecture, following the model of the \acl{jee} architecture: the \textbf{client}, usually a web client or an application client, runs on the client machine. Instead the \textbf{business} code, which is logic that solves or meets the needs of a particular business domain (such as banking) runs on the server machine, as it happens for the \textbf{Web-tier}. The \textbf{\acl{eis}-tier} includes enterprise infrastructure systems, such as the database and legacy systems, and it runs on a third dedicated machine. \\For sake of simpicity, in this document the web client and the mobile application are treated as one entity. For this reason, each communication between server and client will pass through the Web-tier.
\\ \acs{jsf} technology is a server-side component framework for building Java technology-based web applications and we will use it for the dynamic web pages.
As far as the communication with the mobile application is concerned, we will consider an implementation of the \acs{rest} paradigm.\\
\newline
In the following part, we will give a more accurate and detailed decription of this 4-tier architecture.

\begin{itemize}

\item[--]\textbf{Client-tier}: This tier component are the Application Clients and Web Browsers. They both typically have a \acs{gui}, since they interact with the actors.

\item[--]\textbf{Web-tier}: This tier component has the task to manage the requests sent by the Client-tier and to forward these requests to the Business-tier. In a similar way, the Web-tier elaborates the contents generated by the Business-tier and it sends these contents to the Client-tier in a way that this tier components can render the information received.

\item[--]\textbf{Business-tier}: This tier represents the core of the whole system. This tier components contein the logic that solves or meets the needs of a particular business domain such as banking, retail, or finance, is handled by Enterprise Java Beans. They receive data from client programs, processes it, and sends it to the \acs{eis}-tier for storage. An Enterprise Java Bean also retrieves data from storage, processes it, and sends it back to the client program.
All the application logic resides here under the form of Enterprise Java Beans and Java Entities. This tier is connected to the Database through a Java Persistence API.

\item[--]\textbf{\acs{eis}-tier}: This tier components are usually database and legacy systems, where the entire system stores the needed persistent information.

\end{itemize}

\subsection{Component view} \label{subsec:comp-view}

\subsection{Deployment view} \label{subsec:depl-view}

\subsection{Runtime view} \label{run-view}
%You can use sequence diagrams to describe the way components interact to accomplish specific tasks typically related to your use cases

\subsection{Component interfaces} \label{comp-inter}

\subsection{Selected architectural styles and patterns} \label{arch-styles}
%Please explain which styles patterns you used, why, and how

\subsection{Other design decisions} \label{other-des}
