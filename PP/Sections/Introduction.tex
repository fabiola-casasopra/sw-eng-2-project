\section{Introduction} \label{sec:intro}

In the \acl{pp} document, we will show the following information:

\begin{itemize}

\item[\textbf{--}]evaluation of the estimated size of our project;

\item[\textbf{--}]evaluation of the estamated effort and cost;

\item[\textbf{--}]identification of the tasks to be carried out and their schedule;

\item[\textbf{--}]allocation of the available resources to the various tasks previously defined;

\item[\textbf{--}]evaluation of the risks for the project.

\end{itemize}

\subsection{List of Definitions and Abbreviations}
Here there is the acronims and abbreviations list:

%------------Acronyms----------
\begin{acronym}[ACAP] %put here the first acronym

\acro{acap}[ACAP]{Analyst Capability}
\acro{apex}[APEX]{Applications Experience}
\acro{api}[API]{Application Programming Interface}
\acro{cmmi}[CMMI]{Capability Maturity Model Integration}
\acro{cocomo}[COCOMO]{COnstructive COst MOdel}
\acro{cplx}[CPLX]{Product Complexit}
\acro{data}[DATA]{Data Base Size}
\acro{dd}[DD]{Design Document}
\acro{dbms}[DBMS]{DataBase Memory System}
\acro{docu}[DOCU]{Documentation Match to Life-Cycle Needs}
%\acro{gui}[GUI]{Graphical User Interface}
\acro{ei}[EI]{External Input}
\acro{eif}[EIF]{External Interface File}
%\acro{eis}[EIS]{Enterprise Information System}
%\acro{ejb}[EJB]{Enterprise JavaBean}
\acro{eo}[EO]{External Output}
\acro{eq}[EQ]{External Inquiry}
%\acro{er}[ER]{Entity-Relationship}
\acro{flex}[FLEX]{Development Flexibility}
\acro{fp}[FP]{Function Point}
%\acro{gps}[GPS]{Global Positioning System}
\acro{ilf}[ILF]{Internal Logic File}
\acro{itpd}[ITPD]{Integration Test Plan Document}
%\acro{it}[IT]{Integration Test}
%\acro{jdbc}[JDBC]{Java DataBase Connectivity}
\acro{jee}[JEE]{Java Platform, Enterprise Edition}
%\acro{jsf}[JSF]{JavaServer Faces}
\acro{ltex}[LTEX]{Language and Tool Experience}
%\acro{mvc}[MVC]{Model-View-Controller}
\acro{os}[OS]{Operating System}
\acro{pcap}[PCAP]{Programmer Capability}
\acro{pcon}[PCON]{Personnel Continuity}
\acro{plex}[PLEX]{Platform Experience}
\acro{pp}[PP]{Project Plan}
\acro{prec}[PREC]{Precedentedness}
\acro{pmat}[PMAT]{Process Maturity}
\acro{pvol}[PVOL]{Platform Volatility}
\acro{rasd}[RASD]{Requirements Analysis and Specifications Document}
\acro{rely}[RELY]{Required Software Reliability}
\acro{resl}[RESL]{Risk Resolution}
\acro{ruse}[RUSE]{Developed for Reusability}
%\acro{rest}[REST]{Representational State Transfer}
\acro{sced}[SCED]{Required Development Schedule}
\acro{site}[SITE]{Multisite Development}
\acro{sloc}[SLOC]{Source Lines Of Code}
\acro{stor}[STOR]{Main Storage Constraint}
%\acro{tp}[TP]{Test Procedure}
\acro{team}[TEAM]{Team Cohesion}
\acro{time}[TIME]{Execution Time Constraint}
\acro{tool}[TOOL]{Use of Software Tools}
\acro{ufp}[UFP]{Unadjusted Function Poin}
%\acro{ui}[UI]{User Interface}
%\acro{uml}[UML]{Unified Modeling Language}
%\acro{api}[API]{Application Programming Interface}



\end{acronym}

\subsection{List of Reference Documents}

\begin{itemize}
\item[\textbf{--}] Specification document: Assignments AA 2016-2017.pdf
\item[\textbf{--}] \acl{rasd}: RASD.pdf 
\newline
(\url{https://github.com/fabiola-casasopra/sw-eng-2-project/tree/master/RASD/RASD.pdf})
\item[\textbf{--}] \acl{dd}: DD.pdf 
\newline
(\url{https://github.com/fabiola-casasopra/sw-eng-2-project/blob/master/DD/DD.pdf})
\item[\textbf{--}] \acl{itpd}: ITPD.pdf 
\newline
(\url{https://github.com/fabiola-casasopra/sw-eng-2-project/blob/master/ITDP/ITDP.pdf})
\item[\textbf{--}] \url{http://www.qsm.com/resources/function-point-languages-table}: to know the conversion coefficient between \acs{fp}s and \acs{sloc}
\item[\textbf{--}] \url {http://csse.usc.edu/csse/research/COCOMOII/cocomo2000.0/CII_modelman2000.0.pdf}: \acs{cocomo} II model definition manual
\end{itemize}

\subsection{Document overview}

Here we show the stucture of the document, with a brief overview of each section.

\begin{itemize}

\item[\textbf{Section \ref{sec:intro}}]There is an introduction with general information about this document.

\item[\textbf{Section \ref{sec:functpointappr}}]There is the application of Function Points to estimate the size of our project  

\item[\textbf{Section \ref{sec:cocomo}}]There is the application of \acs{cocomo} to estimate effort and cost of our project.

\item[\textbf{Section \ref{sec:psara}}]Here, we are going to identify the tasks to be carried out for our project and their schedule. Moreover, we are going to allocate the available resources to the various tasks.

\item[\textbf{Section \ref{sec:proj-risks}}]Here, we are going to define the risks for the project, their relevance and the associated recovery actions.

\item[\textbf{Section \ref{sec:app}}]Here there are given additional information that may be useful to the reader, such as the tools used and the time spent to redact this document.
\end{itemize}