\section{Introduction} \label{sec:intro}

In the \acl{pp} document, we will show the following information:

\begin{itemize}

\item[\textbf{--}]evaluation of the estimated size of our project;

\item[\textbf{--}]evaluation of the estamated effort and cost;

\item[\textbf{--}]identification of the tasks to be carried out and their schedule;

\item[\textbf{--}]allocation of the available resources to the various tasks previously defined;

\item[\textbf{--}]evaluation of the risks for the project.

\end{itemize}

\subsection{List of Definitions and Abbreviations}
Here there is the acronims and abbreviations list:

%------------Acronyms----------
\begin{acronym}[COCOMO] %put here the first acronym

\acro{cocomo}[COCOMO]{COnstructive COst MOdel}
\acro{dd}[DD]{Design Document}
%\acro{gui}[GUI]{Graphical User Interface}
%\acro{eis}[EIS]{Enterprise Information System}
%\acro{ejb}[EJB]{Enterprise JavaBean}
%\acro{er}[ER]{Entity-Relationship}
%\acro{gps}[GPS]{Global Positioning System}
\acro{itpd}[ITPD]{Integration Test Plan Document}
%\acro{it}[IT]{Integration Test}
%\acro{jdbc}[JDBC]{Java DataBase Connectivity}
%\acro{jee}[JEE]{Java Platform, Enterprise Edition}
%\acro{jsf}[JSF]{JavaServer Faces}
%\acro{mvc}[MVC]{Model-View-Controller}
\acro{pp}[PP]{Project Plan}
\acro{rasd}[RASD]{Requirements Analysis and Specifications Document}
%\acro{rest}[REST]{Representational State Transfer}
%\acro{tp}[TP]{Test Procedure}
%\acro{ui}[UI]{User Interface}
%\acro{uml}[UML]{Unified Modeling Language}
%\acro{api}[API]{Application Programming Interface}
%\acro{db}[DB]{DataBase}
%\acro{os}[OS]{Operating System}

\end{acronym}

\subsection{List of Reference Documents}

\begin{itemize}
\item[\textbf{--}] Specification document: Assignments AA 2016-2017.pdf
\item[\textbf{--}] \acl{rasd}: RASD.pdf 
\newline
(\url{https://github.com/fabiola-casasopra/sw-eng-2-project/tree/master/RASD/RASD.pdf})
\item[\textbf{--}] \acl{dd}: DD.pdf 
\newline
(\url{https://github.com/fabiola-casasopra/sw-eng-2-project/blob/master/DD/DD.pdf})
\item[\textbf{--}] \acl{itpd}: ITPD.pdf 
\newline
(\url{https://github.com/fabiola-casasopra/sw-eng-2-project/blob/master/ITDP/ITDP.pdf})
\end{itemize}

\subsection{Document overview}

Here we show the stucture of the document, with a brief overview of each section.

\begin{itemize}

\item[\textbf{Section \ref{sec:intro}}]There is an introduction with general information about this document.

\item[\textbf{Section \ref{sec:functpointappr}}]There is the application of Function Points to estimate the size of our project  

\item[\textbf{Section \ref{sec:cocomo}}]There is the application of \acs{cocomo} to estimate effort and cost of our project.

\item[\textbf{Section \ref{sec:psara}}]Here, we are going to identify the tasks to be carried out for our project and their schedule. Moreover, we are going to allocate the available resources to the various tasks.

\item[\textbf{Section \ref{sec:proj-risks}}]Here, we are going to define the risks for the project, their relevance and the associated recovery actions.

\item[\textbf{Section \ref{sec:app}}]Here there are given additional information that may be useful to the reader, such as the tools used and the time spent to redact this document.
\end{itemize}