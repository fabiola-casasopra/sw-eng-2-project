\section{Project Risks} \label{sec:proj-risks}
%Define the risks for the project, their relevance and the associated recovery actions

In this section, we are going to define which are the main risks for our project, how relevant they are and which could be an appropriate recovery action.
A risk concerns future happenings, that may involve change in mind, opinion, actions, places, etc., that is everytime we make a choice, we have to deal with the uncertainty that our choice entails.

One of the main risks for our project is related to \textbf{staff size and experience}, that is associated with overall technical and project experience of the software engineers who will do the work. In our case, given the fact that our team is composed by only two people, we have risks related to the availability of all the team member and related to the time contraints of the schedule.
Our team members have the same importance and, moreover, they have similar skills. A problem can be the \textbf{illness of a team member}, maybe even for a long period of time. A way to solve this problem can be that there is more overlap of work and everyone can therefore understand other's jobs. However, it remains the problem related to the schedule: a person alone can't carry out the work of two in the scheduled amount of time. Feasible solution mey be investigate buying-in components or the use of a program generator. Adding other people to the project should not be seen as the primary solution, since it is very difficult the integration of a new team member while the work is already in progress. In order to make it easier and faster, we can focus on the documentation, that should be as much complete and effective as possible.

Another risks that we have to take into considerations is related to our \textbf{development environment} that is associated with availability and quality of the tools to be used to build the project. In our case this could be a seriuos problem, since we have some dependency on external services and components. If there is a change in the terms and conditions of the Google Maps service, or even just a modification of the \acs{api} itself, it could result in serious financial or technical problems. Also, faults in reusable software components have to be repaired before these components are reused and this could lead to significant issues on the financial and business side. A possible solution in order to mitigate this problem is to design the code so that it can be as portable as possible and with a great modularity and independence between components.

Moreover, we have to consider the risk related to the \textbf{process definition}, that is associated with the degree to which the software process has been defined and is followed. In our case, we could have problems related to the project scheduling. In this document, we provide an initial overall schedule; however, we didn't take into account all the possible issues that we can encounter during the different phase of our project. So, a solution could be to allocate some extra time at the predicted end of each major activity, in order to not have strict deadlines and the possibility to make some adjustments. 

Finally, we should not forget to consider the risk related to the \textbf{business impact} that is associated with constraints imposed by management or the marketplace. In our case, the problem coud be the fact that there is concurrency, since there are other systems that offer a solution similar to ours. Here, ther is also the risk related to \textbf{customer characteristics}, that is associated with sophistication of the customer and the developer's ability to communicate with the customer in a timely manner. A way to solve this issue could be to dedicate some effort for marketing, maybe with the help of some experts of this field.